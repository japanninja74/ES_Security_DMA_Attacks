\chapter{Overview on DMA attacks} \label{CH_intro}
\section{Prerequisites}
\subsection{What is the DMA}
Direct Memory Access is a feature that allows different peripheral, to access the memory directly, independently of the CPU. This frees up the CPU to perform other tasks and can result in a significant performance boost for the system. Many hardware systems use DMA, including disk drive controllers, graphics cards and network cards. It is an hardware technology, so it works independently from the operating system, and therefore,  since it is working with physical addresses, it represent a source of non deterministic behaviors, and a possible backdoor for malware attacks.
DMA is typically composed of:
\begin{itemize}
    \item  A DMA controller: responsible for coordinating the transfer of data between memory and peripheral devices.
\item A set of DMA channels: allows multiple devices to perform DMA transfers simultaneously.
\item A DMA buffer: a small amount of memory reserved for use by the DMA controller to temporarily store data during transfers.
\end{itemize}

We can have two different organization of the DMA engine:
\begin{itemize}
    \item Third-party DMA: The host CPU configures (source and destination
address) the central DMA controller to perform a DMA transfer. The DMA controller interrupts the host CPU when the DMA transfer has been finished.
Hence, the host CPU is aware of a third-party DMA transfer. 
\item First-party DMA: The peripheral device can configure its own DMA engine. The device acts as bus master to get control of the system bus to perform a DMA transfer. The device can interrupt the host CPU when the device has completed the transfer. The transfer also works if the device does not interrupt the host CPU at the end of the DMA transfer. In this case the CPU is completely unaware of the DMA transfer.
\end{itemize}


\subsection{Memory protection}

Memory protection is a crucial aspect of computer systems, especially for modern operating systems, as it helps prevent errors and security breaches. Instead of using the real addresses of the memory space, we use \textit{virtual addresses} which will be translated to the real ones by proper hardware devices. Since there is always an entity (called MMU - Memory Management Unit) between us and the memory, it is impossible to access freely the memory. It can be implemented in different ways, such as:
\begin{itemize}
    \item Memory segmentation: This technique divides memory into distinct segments, each with its own specific purpose and access control permissions. This helps prevent programs from accessing memory locations they are not authorized to access.

    \item Paging: Paging is a technique where the physical memory is divided into fixed-sized blocks called pages, which can be individually assigned to different processes. This allows the operating system to control and monitor the memory access of each process.

    \item Access control: Access control is a mechanism that defines which processes or users are allowed to access specific memory locations, and what actions they are permitted to perform. This helps prevent unauthorized access to sensitive information.

    \item Virtual memory: Virtual memory is a technique that allows a process to use more memory than physically available by temporarily transferring data from physical memory to a disk. This helps ensure that processes don't interfere with each other's memory usage.

\end{itemize}
The implementation can vary depending on the operating system, hardware, and security requirements. 

% I removed all the paragrahps realted to IOMMU since it is already covered later on 

In spite of its many advantages, in many embedded systems this implementation is avoided to reduce costs, complexity and time to market. 

\section{An introduction to DMA attacks} \label{intro}
DMA attacks are a particularly powerful class of attacks, especially for embedded systems. These attacks enable a potential attacker to read and write memory off a victim system directly, bypassing the main CPU and OS. Achieving DMA attacks in embedded systems is easier than in a general purpose computer, due to the absence of memory protection, and more dangerous because often flat architecture operating systems are implemented in those kind of systems, which allow the attacker to access reserved and critical areas, like the ones of the OS itself.

By overwriting memory in a DMA transfer, for instance just increasing the size of the burst transfer, the attacker can modify and control reserved memory regions, performing any manner of malicious activity.

Direct Memory Access is a capability implemented in every system which has the need of speeding up the transfer from/to peripherals. It permits to execute the transfer without the aid of neither CPU nor Operating System, after  setting up the DMA, so that the peripherals can directly communicate with the system's memory and save time.

Despite the possibility to accelerate the system, DMA exposes the system to various attacks, for instance:

\begin{itemize}
    \item Access and/or modify reserved information;
    \item Extend control over the execution of the kernel itself;
    \item Malware injection.
\end{itemize}



\section{Some examples of DMA attacks}
One of the main tools used to perform DMA attacks is \href{https://github.com/ufrisk/pcileech}{pcileech}. Most of the works you can find on the internet regarding DMA attacks are proof of concept, though there are some cases in which hackers were able to inject malicious code inside a computer once they had access to the hardware. 

\section{Countermeasures}

DMA attacks can be difficult to prevent because they occur at a hardware level and are not visible to the operating system or CPU. However, there are some measures that can be taken to reduce the risk of DMA attacks and make them more difficult to execute.

One way to prevent DMA attacks is to use an IOMMU (Input/Output Memory Management Unit). An IOMMU is a hardware component that acts as a memory management unit for input/output (I/O) operations. It allows the system to restrict the memory addresses that I/O devices can access, making it more difficult for an attacker to perform a DMA attack.

A different approach is to use hardware-enforced memory isolation. This technique involves dividing the system's memory into different regions, with each region assigned a specific level of access. This can help to prevent an attacker from gaining access to sensitive areas of memory.

Another measure that can be considered to prevent DMA attacks is to use secure boot. Secure boot is a process that verifies the integrity of the system's boot process. It ensures that only trusted software is allowed to run when the system boots up, making it more difficult for an attacker to inject malicious code into the system.

Other approaches to preventing DMA attacks include using hardware-based firewalls, implementing role-based access control, and using hardware-based encryption.


