\chapter{Documentation} \label{CH_doc}
In this chapter the developed project simulating the DMA Attack through UART peripheral will be described in details.
All the code developed is intended to be used with the microprocessor LPC1768. The user manual for this microprocessor can be found \href{https://www.keil.com/dd/docs/datashts/philips/lpc17xx_um.pdf}{here}. This was the main source for the project: as it is known, firmware developement mainly consists in understanding and applying concepts from it, with the aim of a working system. Other interesting sources to write our code were \href{https://community.nxp.com/t5/LPC-Microcontrollers/LPC1768-UART-receive-using-DMA/m-p/800746}{this} and  \href{https://www.cartagena99.com/recursos/alumnos/apuntes/Tema%202.%20Perifericos%20en%20el%20LPC1768%20(ppt).pdf}{this} (though part of this presentation is in spanish). 

The code is meant to be clear and modular, through the use of macros.

\section{UART Library}
The following subsections describe the composition of the UART library and how to enable the peripheral for communication (both sending and receiving). This library can be used with peripherals 0, 2, and 3. UART 1 has a different structure.

\subsection{uart\_parameters}

\begin{lstlisting}[language= C, caption = Structure definition for UART parameters]
struct uart_parameters{
	uint32_t	uart_n;
	uint32_t	clock_selection;
	uint32_t	DLL;
	uint32_t	DLM;
	uint32_t	DivAddVal;
	uint32_t	MulVal;
	uint32_t	fifo_enable;
	uint32_t	dma_mode;
	uint32_t	trigger_level;
	uint32_t	rdv_int;
	uint32_t	word_size;
	uint32_t	stop_bit_n;
	uint32_t	parity_active;
	uint32_t	parity_type;
};
\end{lstlisting} \label{uart_struct}

This structure is used to define the communication parameters. Each field of the structure can take either a numeric value or a specific value as specified in the LPC1768 user manual. To make it easier for the user to use the library without needing to know all the low-level details, we provide many macros with the different values.

\begin{itemize}
    \item \texttt{uart\_n}: this value selects which of the three available UART peripheral you want to use. Possible values are \texttt{UART0}, \texttt{UART1} and \texttt{UART3}. 
    \item \texttt{clock\_selection}: these values determine which clock frequency the peripheral uses, selected from four different submultiples of the CPU clock (\texttt{CCLK}): \texttt{UART\_CCLK\_DIV\_4} (CCLK / 4),\texttt{UART\_CCLK\_DIV\_1} (CCLK) ,\texttt{UART\_CCLK\_DIV\_2} (CCLK / 2), and \texttt{UART\_CCLK\_DIV\_8} (CCLK / 8).
    \item \texttt{DLL}, \texttt{DLM}, \texttt{DivAddVal}, \texttt{MulVal}: these four values are used to set the baud rate for the UART connection, as explained in the LPC1768 user manual.
    \item \texttt{fifo\_enable}: this value potentially enables the FIFO mode, which is required to use DMA. Its values are \texttt{UART\_EN\_FIFO} and \texttt{UART\_NO\_FIFO}.
    \item \texttt{dma\_mode}: this value potentially enables the DMA mode. Its values are \texttt{UART\_DMA} and \texttt{UART\_NO\_DMA}.
    \item \texttt{trigger\_level}: the trigger level is the number of elements that need to be received before an interrupt is raised and in case the DMA is enabled, it will trigger a DMA request interrupt, that will result into a DMA transfer. Possible values are \texttt{UART\_TR\_LEVEL\_1} (trigger level 1), \texttt{UART\_TR\_LEVEL\_4} (trigger level 4), \texttt{UART\_TR\_LEVEL\_8} (trigger level 8), and \texttt{UART\_TR\_LEVEL\_16} (trigger level 16).
    \item \texttt{rdv\_int}: with this value, we can enable an interrupt when new data is received. Possible values are \texttt{UART\_RECV\_DV\_INT} and \texttt{UART\_NO\_INT};
    \item \texttt{word\_size}: this value determines the size of a word in the communication; possible values are \texttt{UART\_WORD\_5} (5), \texttt{UART\_WORD\_6} (6), \texttt{UART\_WORD\_7} (7) and \texttt{UART\_WORD\_8} (8).
    \item \texttt{stop\_bit\_n}: this value determines the number of stop bits we want, between 1 (\texttt{UART\_STOP\_1} and 2 (\texttt{UART\_STOP\_2}). 
    \item \texttt{parity\_active}: this element allows to enable (\texttt{UART\_PARITY}) or disable (\texttt{UART\_NO\_PARITY}) the parity field in the transmission.
    \item \texttt{parity\_type}: this determines the type of parity we want, between odd (\texttt{UART\_ODD\_PARITY}) and even (\texttt{UART\_EVEN\_PARITY}). 
\end{itemize}
\subsection{init\_uart}
This function takes a pointer to a \texttt{uart\_parameters} structure as an argument and initializes the relevant registers with the provided values. It is also responsible for powering on the peripheral through the register \texttt{PCONP}. 

\section{DMA Library}
\subsection{dma\_transfer}

\begin{lstlisting}[language= C, caption = DMA transfer struct]
struct dma_transfer{
	uint32_t	channel;
	uint32_t 	size;
	uint32_t 	dest;
	uint32_t 	source;
	uint32_t	source_burst_size;
	uint32_t	dest_burst_size;
	uint32_t 	source_increment;
	uint32_t 	dest_increment;
	uint32_t	source_transfer_width;
	uint32_t	dest_transfer_width;
	uint32_t	terminal_count_enable;
	uint32_t 	transfer_type;
	uint32_t 	source_peripheral;
	uint32_t	dest_peripheral;
	uint32_t 	lli_head;
};
\end{lstlisting} \label{dma_struct}

Just like with the UART definition, each value can either be a numeric value or a specific value expressed through a macro.

\begin{itemize}
    \item \texttt{channel}: this value determines which channel we want to use. There are 8 channels available in the GPDMA, numbered from 0 to 7, with the lower numbers having higher priority. The macros \texttt{DMA\_CHANNEL\_x} can be used to fill this field.
    \item \texttt{size}: how many elements we want to transfer.
    \item \texttt{dest}: address of the destination.
    \item \texttt{source}: address of the source.
    \item \texttt{source\_burst\_size}, \texttt{dest\_burst\_size}: the burst size refers to the number of data items that are transferred in a single burst during a DMA transfer. It is typically measured in the number of items transferred per DMA transfer request. For example, a burst size of 8 means that 8 data items are transferred per DMA request. The burst size can affect the performance of the DMA transfer and may need to be optimized for the specific application. The values these two fields can assume are all the power of 2 between 1 and 256, using the macros \texttt{DMA\_BURST\_SIZE\_x}. 
    \item \texttt{source\_increment}, \texttt{dest\_increment}: these values determine if, after each transfer, we want to increment the source and destination address. The possible values are \texttt{DMA\_NO\_INCREMENT} and \texttt{DMA\_INCREMENT}. 
    \item \texttt{source\_transfer\_width}, \texttt{dest\_transfer\_width}: width of each element of source and destination. Possible values are 8 (byte), 16 (half word) and 32 (word), expressed through the macros \texttt{DMA\_TRANSFER\_WIDTH\_8}, \texttt{DMA\_TRANSFER\_WIDTH\_16} and \texttt{DMA\_TRANSFER\_WIDTH\_32}.
    \item \texttt{terminal\_count\_enable}: with the value \texttt{DMA\_TERMINAL\_COUNT\_ENABLE} we generate an interrupt at the end of the transaction, otherwise (using \texttt{DMA\_TERMINAL\_COUNT\_NOT\_ENABLE}) no.
    \item \texttt{transfer\_type}: there are 4 kinds of transfer: memory to memory (\texttt{DMA\_M2M}), memory to peripheral (\texttt{DMA\_M2P}), peripheral to memory (\texttt{DMA\_P2M}) and peripheral to peripheral (\texttt{DMA\_P2P}). 
    \item \texttt{source\_peripheral}, \texttt{dest\_peripheral}: these fields determine which are the source or destination peripheral of the transfer. This field is used to determine when to initiate a transaction based on the status of the peripheral. Given the large number of peripherals, we have not defined a macro for each one. Possible values can be found on page 592 of the user manual.
    \item \texttt{lli\_head}: if we want to use DMA with the scatter gather technique, we fill this field with the pointer to the head of the linked list (whose structure is defined in \texttt{dma\_lli\_element}). 
\end{itemize}

\subsection{dma\_lli\_element}
\begin{lstlisting}[language= C, caption = DMA transfer struct]
struct dma_lli_element{
	uint32_t 					src;
	uint32_t 					dest;
	struct dma_lli_element* 	next_lli;
	uint32_t					ctrl;
};
\end{lstlisting} \label{dma_lli_struct}
This structure is used to perform a scatter gather operation. 

\subsection{init\_DMA}
This function activates the GPDMA through the register \texttt{PCONP} and enables the associated interrupt.

\subsection{transfer\_DMA}
This function uses a pointer to \texttt{dma\_struct} as argument and initialize a DMA transaction. 


\section{Application}

An application based on Micrium OS has been developed, in order to test the libraries and simulate the DMA attack on a real system, in this specific case the target board is the LandTiger V2.0 NXP LPC1768 ARM development board.

The simulated scenario of the application is a DMA attack to the control unit charged to monitor the state of the payload system in a spacecraft. The aim is to cause the failure of the mission, changing willingly the parameters of the sensors to shut down the on board computer OBC, which cannot operate over certain values of pression and temperature.

\subsection{Setup}
The setup used for the development of the application has been the following:
\begin{itemize}
    \item LandTiger V2.0 NXP LPC1768 ARM development board
    \item Keil uVision 5  running on a General Purpose PC
    \item Putty terminal emulator open source
    \item Uart to USB cable 
\end{itemize}

\subsection{Micrium OS}
The application is built above the operating system Micrium, which is preemptive, deterministic, royalty-free and based on a flat architecture. 

As already discussed in \ref{intro}, DMA attacks are even easier to perform on these kind of architectures, because there is no distinct division between application and operating system, so it is possible to access reserved memory area, through the application.

Following the Task Model typical of Micrium, few tasks have been created and relative TCB and Stack have been allocated. These are:
\begin{itemize}
    \item APP task 
    \item Lock task
    \item Display task
\end{itemize}

\subsection{APP task}
The purpose of the application is to communicate with sensors and take actions depending on the data, so it is required to set-up UART communication and DMA transfer. This task has the highest priority.

In order to initialize a UART comunication, a struct \texttt{uart\_parameters} has to be defined and instantiated. Here all the parameters are set, in particular the most important ones are:
\begin{itemize}
    \item \texttt{Clock frequency} : (CPU Clock)/4
    \item \texttt{Baud rate}: 9645 (Calculated using algorithm explained in user manual)
    \item \texttt{DMA mode}: ON
    \item \texttt{Trigger level}: 1 (as soon as an element arrives a DMA request is sent)
    \item \texttt{Word size}: 8 bits (1 Byte)
\end{itemize}

After the initialization of the UART connection the task is suspended and will be resumed by the \texttt{lock\_task}.

As soon as \texttt{app\_task} is resumed by \texttt{lock\_task}, it initializes the DMA transfer, setting all the required parameters in an endless loop. The key ones are:
\begin{itemize}
    \item \texttt{Destination}: Struct handling the data coming from the sensor
    \item \texttt{Source}: Receiver Buffer Register of UART
    \item \texttt{Burst size Dest/Source}: 1 (Just transfer one data item at a time)
    \item \texttt{Terminal Count Enable}: ON (Generate interrupt at the end of the transfer)
    \item \texttt{Transfer Type}: Peripheral to Memory (UART to memory)
\end{itemize}
After the parameters are set up,  \texttt{transfer\_DMA} function is called and the DMA transfer is initialized. Then the task is suspended again. 


\subsection{Lock task}
Data coming from the sensor has a CRC field which is calculated before sending the data to the board. \texttt{Lock task} recalculates the CRC and compares it to the received one, to make sure that the data was sent correctly. 

If the data is accepted, which means the calculated CRC corresponds to the received one, the condition to open the door is checked, updating the related flag. Depending on its final value the door is opened or closed and the state is saved. 

The \texttt{App\_task} is then resumed and the \texttt{Lock\_task} is suspended. 

\subsection{Display task}
This task has the lowest priority among the others, and it is responsible to display on screen the data coming from the sensor. 


\subsection{DMA\_IRQHandler}
This DMA handler is called when the DMA transfer is completed. 
The door's flag is reset and the LCD is cleared.

Depending on the result of the redundancy check, the data might be taken or not. If it is, no action is performed because Lock\_task already did it, but if the CRC condition is not satisfied the door's state still needs to be decided and this is done restoring previous state.


Finally it resumes the \texttt{Lock\_task}, restarting the flow of the application. 